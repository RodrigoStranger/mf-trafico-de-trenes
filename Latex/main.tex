\documentclass[journal]{IEEEtran}
\usepackage[utf8]{inputenc} % Codificación de entrada UTF-8
\usepackage{cite} % Para citaciones
\usepackage{placeins} % Para \FloatBarrier
\usepackage{graphicx} % Para incluir imágenes
\usepackage{float}
\usepackage{amsmath} % Para fórmulas matemáticas
\usepackage{url} % Para manejar URLs
\usepackage{listings}
\usepackage{xcolor}
\usepackage[ruled,vlined]{algorithm2e} % Para pseudocódigo
\usepackage{authblk}
\usepackage{hyperref}
\usepackage{breakurl}
\usepackage[spanish]{babel}

% Configuración de listings para bash
\lstset{
  language=bash,
  basicstyle=\ttfamily, 
  columns=fullflexible, 
  frame=single, 
  breaklines=true, 
  postbreak=\mbox{\textcolor{red}{$\hookrightarrow$}\space}, 
  backgroundcolor=\color{lightgray!20},  
  keywordstyle=\color{blue}, 
  commentstyle=\color{green}, 
  stringstyle=\color{red} 
}

\title{Especificación Formal y Validación para la Prevención de Colisiones en Tráfico Ferroviario}
\author[1]{Infanzón Acosta R. E.}
\author[1]{Aguilar Chirinos C. D.}
\author[1]{Quispe Huanca A. M.}
\affil{Universidad La Salle, Arequipa, Perú}



\author[2]{Molina Barriga M. - Mentor }

\begin{document} 

\maketitle

\begin{abstract}
Aquí el Abtract o resumen
\end{abstract}

\begin{IEEEkeywords}
Aquí las Keywords
\end{IEEEkeywords}

\section{Introducción}
La seguridad en el transporte en trenes es una meta clave para cuidar el bienestar de mucha gente que usa este tipo de transporte a diario. Con el tiempo, los nuevos inventos tecnológicos han hecho que los trenes sean cada vez más fáciles y rápidos de usar, pero aun así los accidentes y choques en las vías siguen siendo un peligro importante. Las malas consecuencias de estos sucedidos no solo afectan a los pasajeros y trabajadores sino que también tienen un efecto grande en las comunidades y economías que dependen del tren. En este marco, usar nuevas ideas para mejorar la forma de evitar que los trenes choquen es un gran paso hacia un sistema más seguro.

El avance de máquinas que encuentran y avisan sobre problemas del tren es una parte importante de la ingeniería de hoy. La idea principal de este plan es usar un mecanismo basado en tecnologías nuevas que pueda parar choques bien. Aún cuando las vías de tren han mejorado mucho en cuanto a estructura y manejo, lo difícil es reducir los equivocaciones de las personas y las fallas en la comunicación entre partes del sistema. Aquí , la tecnología importa mucho porque pone acción automática en tareas claves, como ver trenes en las vías, controlar cuán rápido van y actuar frente a problemas.

\begin{itemize}
    \item \textbf{VDM++:} Se utilizará para modelar formalmente los componentes del sistema de prevención de colisiones, como trenes, sensores y semáforos. VDM++ permitirá crear una representación matemática precisa de las interacciones y verificar la validez de las decisiones, como cambios de señales o ajustes de velocidad, garantizando su correcta ejecución en todas las condiciones.
    \item \textbf{Model-Checking (M\'aquina de estados)} Este método verificará el funcionamiento del sistema en todos sus posibles estados y transiciones. Utilizando model-checking, se asegurará que las acciones, como la activación de semáforos o el frenado de los trenes, se realicen correctamente, minimizando los riesgos de error y garantizando la seguridad en cada proceso.
\end{itemize}


\subsection{Alcance esperado}
El alcance de este proyecto se centra en la aplicación de métodos formales para estructurar y analizar el sistema de prevención de colisiones ferroviarias. A través del uso de VDM++ y model-checking, se busca desarrollar un modelo formal que permita verificar la consistencia y validez de las interacciones entre los componentes del sistema, como los trenes, sensores y semáforos. Este enfoque garantizará la integridad de las decisiones automatizadas, como los cambios de señales o ajustes de velocidad, y asegurará que el sistema funcione correctamente en todas las condiciones posibles. Mediante un análisis riguroso y sistemático, se contribuirá a mejorar la seguridad del tráfico ferroviario, proporcionando una base sólida para el desarrollo de herramientas futuras que optimicen la prevención de accidentes en el ámbito ferroviario.


\section{Resumen Ejecutivo}
\subsection{Antecedentes}
\subsubsection{Un Sistema de Seguridad y Prevención de Colisiones en Tiempo Real para Trenes}
\subsubsection*{Resumen}
El artículo presenta un sistema integral de seguridad que combina diversas tecnologías avanzadas para la prevención de colisiones en el transporte ferroviario. Reconoce que, a pesar de las mejoras en la infraestructura y las técnicas de gestión, los peligros persisten, lo que justifica la implementación de un enfoque más proactivo. El sistema propuesto utiliza datos en tiempo real para detectar y responder a situaciones de riesgo, con el fin de prevenir accidentes antes de que ocurran. Además, se subraya la importancia de la comunicación entre trenes, estaciones y otros elementos del sistema para asegurar una operación sincronizada y segura.
\subsubsection*{Puntos de interés para la investigación}
\begin{itemize}
    \item \textbf{Detección y Respuesta en Tiempo Real:}
    El sistema se basa en tecnologías de detección que permiten identificar situaciones peligrosas en tiempo real, activando respuestas automáticas que pueden incluir la reducción de velocidad o la detención completa antes de que se produzca una colisión\cite{railwaycouncil2017realsafetysystem}.
    \item \textbf{Integración de Tecnologías Avanzadas:}
    Se integran tecnologías como el monitoreo por GPS, sensores de proximidad y sistemas de comunicación entre trenes (CBTC), lo que aumenta la conciencia situacional y la capacidad de respuesta ante emergencias\cite{railwaycouncil2017realsafetysystem}.
    \item \textbf{Comunicaciones Eficientes:}
    La comunicación efectiva entre los trenes y las estaciones es fundamental para coordinar acciones y minimizar riesgos. El artículo destaca la necesidad de protocolos de comunicación robustos para asegurar que la información crítica se transmita sin demora\cite{railwaycouncil2017realsafetysystem}.
    \item \textbf{Beneficios Socioeconómicos:}
    Al reducir la frecuencia y severidad de los accidentes, el sistema no solo mejora la seguridad de los pasajeros y el personal, sino que también se traduce en beneficios económicos para las comunidades que dependen del transporte ferroviario, mitigando pérdidas asociadas a accidentes y paradas operativas\cite{railwaycouncil2017realsafetysystem}.
    \item \textbf{Evaluación Continua y Mejora del Sistema:}
    El sistema propuesto incluye mecanismos para la evaluación continua y la mejora de los procedimientos de seguridad, asegurando que se mantenga actualizado frente a nuevas amenazas y desarrollos tecnológicos en el ámbito ferroviario\cite{railwaycouncil2017realsafetysystem}.
\end{itemize}

\subsubsection{SafeCap: Un Sistema de Seguridad para la Prevención de Colisiones en Tráfico Ferroviario}
\subsubsection*{Resumen}
El documento presenta el sistema SafeCap, diseñado específicamente para abordar la seguridad en las operaciones ferroviarias y evitar colisiones entre trenes. A pesar de las mejoras en la infraestructura ferroviaria y las tecnologías actuales, los accidentes siguen siendo una preocupación significativa. SafeCap utiliza sensores y tecnología de comunicación para proporcionar un monitoreo continuo del estado de los trenes y su entorno. El sistema puede detectar situaciones potencialmente peligrosas y activar alertas o intervenciones automáticas para prevenir colisiones.
\subsubsection*{Puntos de interés para la investigación}
\begin{itemize}
    \item \textbf{Detección Proactiva de Riesgos:}
    SafeCap implementa tecnologías de detección avanzada que permiten identificar riesgos en tiempo real, lo que facilita una respuesta rápida ante situaciones peligrosas y atiende a la seguridad de los viajeros y el personal\cite{ada2014safecap}.
    \item \textbf{Integración de Sensores y Sistemas de Comunicación:}
    El sistema hace uso de diversos sensores que recopilan datos críticos y los comunica a través de redes seguras, asegurando que toda la información relevante esté disponible para el análisis inmediato, lo que mejora la toma de decisiones en situaciones de emergencia\cite{ada2014safecap}.
    \item \textbf{Intervenciones Automáticas:}
    Una característica clave de SafeCap es la capacidad de llevar a cabo intervenciones automáticas, como frenar o desviar trenes, a fin de evitar accidentes, lo que reduce la dependencia de la intervención humana y el margen de error asociado\cite{ada2014safecap}.
    \item \textbf{Mejora Continua del Sistema:}
    Se incorpora un enfoque de mejora continua donde el sistema aprende de incidentes previos y se ajusta para optimizar su rendimiento, asegurando que se mantenga actualizado frente a nuevas amenazas y se mejore la efectividad de la prevención de colisiones\cite{ada2014safecap}.
    \item \textbf{Impacto Social y Económico:}
    La implementación de SafeCap no solo incrementa la seguridad en el transporte ferroviario, sino que también tiene un impacto positivo en las comunidades, al reducir los costos económicos relacionados con accidentes y mejorar la confianza del público en el sistema ferroviario\cite{ada2014safecap}.
\end{itemize}

\subsubsection{Intel: Sistemas de Prevención de Colisiones en Trenes}
\subsubsection*{Resumen}
El informe detalla la importancia de los sistemas de prevención de colisiones en el contexto del transporte ferroviario. A pesar de que el medio ferroviario es considerado uno de los más seguros, los accidentes continúan representando una amenaza significativa. Los sistemas de prevención de colisiones utilizan tecnología de monitoreo y comunicación para identificar y mitigar riesgos en tiempo real, permitiendo la intervención automática ante situaciones de peligro. Esto no solo ayuda a evitar accidentes, sino que también optimiza las operaciones del servicio ferroviario.
\subsubsection*{Puntos de interés para la investigación}
\begin{itemize}
    \item \textbf{Monitoreo en Tiempo Real:}
    Los sistemas de prevención de colisiones se basan en tecnologías de monitoreo en tiempo real, que permiten la evaluación constante de las condiciones operativas y la detección de situaciones de riesgo, garantizando respuestas rápidas\cite{intel2022collision}.
    \item \textbf{Interconexión de Sistemas:}
    Se destaca la importancia de la interconexión entre diferentes componentes del sistema ferroviario, como señales, trenes y centros de control, para una comunicación efectiva. Esta integración es esencial para la coordinación de acciones preventivas\cite{intel2022collision}.
    \item \textbf{Intervenciones Automáticas:}
    Los sistemas son capaces de implementar intervenciones automáticas, como el frenado o cambios de ruta, cuando se detectan condiciones peligrosas, reduciendo la dependencia del operador humano y minimizando el margen de error\cite{intel2022collision}.
    \item \textbf{Compatibilidad con Nuevas Tecnologías:}
    El informe resalta cómo los sistemas de prevención de colisiones se pueden integrar con tecnologías emergentes, como inteligencia artificial y análisis de datos, para mejorar aún más su eficacia y adaptabilidad frente a futuros desafíos\cite{intel2022collision}.
    \item \textbf{Beneficios Económicos y Sociales:}
    La reducción de accidentes no solo tiene un impacto positivo en la seguridad de los usuarios, sino que también contribuye a la eficiencia económica del transporte ferroviario, disminuyendo costos asociados a paradas no planificadas y daños. Un sistema ferroviario más seguro fomenta la confianza pública en los viajes en tren, beneficiando a las comunidades y economías locales\cite{intel2022collision}.
\end{itemize}

\section{Requerimientos funcionales}

\subsubsection{Detección de ocupación de vías} 
El sistema debe ser capaz de detectar de manera precisa si una vía está ocupada por un tren u objeto, y debe activar automáticamente los semáforos correspondientes para evitar que un tren entre en una vía ocupada.

\subsubsection{Cálculo de distancia entre trenes} 
El sistema debe medir continuamente la distancia entre trenes en circulación para garantizar que se mantenga una distancia de seguridad adecuada, y debe alertar o activar mecanismos de frenado si la distancia se reduce a niveles peligrosos.

\subsubsection{Ajuste automático de velocidad} 
El sistema debe ser capaz de ajustar la velocidad de los trenes en función de las condiciones de la vía, como la proximidad a otros trenes, la ocupación de las vías y otros factores relevantes, para garantizar que los trenes puedan frenar de manera segura en situaciones críticas.

\subsubsection{Control de la integridad de la vía} 
El sistema debe supervisar continuamente el estado de la infraestructura ferroviaria, incluyendo las vías, para detectar fallos estructurales o desviaciones que puedan comprometer la seguridad, y activar medidas preventivas en caso de identificar un problema.

\subsubsection{Verificación en tiempo real} 
El sistema debe ser capaz de verificar en tiempo real las decisiones tomadas por los sensores y los semáforos, las acciones que se basen en datos correctos y precisos, y proporcionar informes de error si se detectan inconsistencias o fallos en el proceso.


\bibliographystyle{IEEEtran}
\sloppy
\bibliography{citas} % Nombre del archivo .bib donde tienes las referencias

\end{document}
